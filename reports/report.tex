\documentclass[11pt,a4paper]{article}
\usepackage[utf8]{inputenc}
\usepackage[T1]{fontenc}
\usepackage[french]{babel}
\usepackage{lmodern}
\usepackage{geometry}
\geometry{margin=2.3cm}
\usepackage{amsmath,amssymb}
\usepackage{graphicx}
\usepackage{booktabs}
\usepackage{hyperref}

\title{Projet Spark : Analyse et Prévision de la Pollution Urbaine}
\author{Nom Prénom}
\date{Année 2025--2026}

\begin{document}
\maketitle

\section{Contexte et objectifs}
L'objectif est de concevoir une application capable de mesurer, analyser et prédire la pollution dans un réseau urbain, en utilisant Apache Spark et la programmation fonctionnelle (RDD/DataFrames/SQL), avec un volet GraphX et un volet MLlib. % sujet
\par Les mesures considérées : CO\textsubscript{2}, PM2.5, bruit, humidité et des variables temporelles/contexte (stations, lignes, trafic, événements, jours fériés). % sujet

\section{Données}
Deux choix sont possibles : données publiques ou dataset simulé. Dans ce projet, nous générons un dataset simulé cohérent :
\begin{itemize}
  \item \texttt{stations.csv} : identifiant, coordonnées, zone, type de transport
  \item \texttt{edges.csv} : connexions entre stations (distance, ligne)
  \item \texttt{measurements.csv} : mesures horaires par station (CO2, PM2.5, bruit, etc.)
\end{itemize}

\section{Ingestion et préparation}
Lecture CSV avec schéma explicite, suppression des doublons (\texttt{ts, station\_id}) et gestion des valeurs manquantes.
Les variables temporelles \texttt{hour}, \texttt{day\_of\_week}, \texttt{month} et \texttt{is\_peak\_hour} sont extraites.

\section{Exploration et analyses}
\subsection{Statistiques station/ligne}
Calculs : moyenne, min, max, percentile 95\% par station ; et statistiques par ligne.

\subsection{Pics horaires}
Agrégation par station et heure pour identifier les périodes critiques.

\subsection{Indicateur global de pollution}
On définit un indicateur global (normalisé par station) :
\[
PI = 0.45\,\tilde{PM2.5} + 0.35\,\tilde{CO2} + 0.15\,\tilde{Bruit} + 0.05\,\tilde{Humidité}
\]
où $\tilde{x}$ est une normalisation min-max par station.

\subsection{Anomalies}
Détection automatique via Z-score (par station) sur PM2.5 et sur l'indice global :
\[
z = \frac{x - \mu}{\sigma}
\]
Une anomalie est signalée si $|z| > 3$.

\section{Modélisation graphe (GraphX)}
Les stations sont des sommets, les connexions des arêtes. Nous calculons :
\begin{itemize}
  \item centralité simple (degré),
  \item PageRank,
  \item propagation simplifiée de la pollution par diffusion itérative.
\end{itemize}

\section{Prédiction (MLlib)}
On construit un dataset supervisé : prédire $PM2.5(t+1)$ à partir de variables temporelles, contextuelles et de lags (\texttt{pm25\_lag1, pi\_lag1}). Deux modèles sont évalués :
\begin{itemize}
  \item Random Forest Regressor,
  \item Gradient-Boosted Trees Regressor.
\end{itemize}
Métriques : RMSE, MAE, $R^2$.

\section{Conclusion et limites}
Le pipeline couvre ingestion, transformations fonctionnelles, analyses, graphe et prédiction. Extensions possibles : streaming complet, données météo réelles, modèle de propagation plus réaliste (vent/flux), validation temporelle stricte.

\end{document}
