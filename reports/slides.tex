\documentclass{beamer}
\usetheme{Madrid}
\usepackage[utf8]{inputenc}
\usepackage[T1]{fontenc}
\usepackage[french]{babel}

\title{Projet Spark : Analyse \& Prévision de la Pollution Urbaine}
\author{Nom Prénom}
\date{2025--2026}

\begin{document}

\begin{frame}
  \titlepage
\end{frame}

\begin{frame}{Objectifs}
\begin{itemize}
  \item Traitement Big Data : RDD, DataFrames, Spark SQL
  \item Programmation fonctionnelle : map, filter, flatMap, reduce, aggregate
  \item Analyses : stations les plus exposées, pics horaires, indicateur global, anomalies
  \item GraphX : relations stations, centralité, propagation
  \item MLlib : pipeline de prédiction (régression)
  \item Streaming (optionnel) : flux capteurs + anomalies
\end{itemize}
\end{frame}

\begin{frame}{Données}
\begin{itemize}
  \item Mesures : CO2, PM2.5, bruit, humidité, température
  \item Variables : horaires/jours/mois, trafic, événements, jours fériés
  \item Tables : \texttt{stations.csv}, \texttt{edges.csv}, \texttt{measurements.csv}
\end{itemize}
\end{frame}

\begin{frame}{Ingestion \& Nettoyage}
\begin{itemize}
  \item Lecture CSV avec schéma explicite
  \item Suppression doublons : \texttt{dropDuplicates(ts, station\_id)}
  \item Valeurs manquantes : \texttt{na.drop}, \texttt{na.fill}
  \item Parsing timestamp : \texttt{to\_timestamp}
\end{itemize}
\end{frame}

\begin{frame}{Transformations fonctionnelles}
\begin{itemize}
  \item DataFrames : \texttt{withColumn}, \texttt{groupBy().agg()}
  \item RDD : \texttt{map/filter/flatMap/reduceByKey} (ex. comptage anomalies)
  \item Features temporelles : hour, day\_of\_week, month, peak-hour
\end{itemize}
\end{frame}

\begin{frame}{Analyses}
\begin{itemize}
  \item Top stations polluées (moyenne, max, p95)
  \item Pics horaires et périodes critiques
  \item Indicateur global (min-max + pondérations)
  \item Anomalies : Z-score ($|z|>3$)
\end{itemize}
\end{frame}

\begin{frame}{GraphX}
\begin{itemize}
  \item Sommets = stations, arêtes = connexions (distance/ligne)
  \item Mesures : degré, PageRank
  \item Propagation : diffusion itérative de l'indice global
\end{itemize}
\end{frame}

\begin{frame}{Prédiction (MLlib)}
\begin{itemize}
  \item Objectif : prédire $PM2.5(t+1)$
  \item Features : temps + contexte + lags (\texttt{pm25\_lag1, pi\_lag1})
  \item Modèles : Random Forest, Gradient Boosted Trees
  \item Évaluation : RMSE, MAE, $R^2$
\end{itemize}
\end{frame}

\begin{frame}{Streaming (optionnel)}
\begin{itemize}
  \item Simulation capteurs : fichiers déposés dans un dossier
  \item \texttt{readStream} + transformations + alerte anomalies
  \item Sortie : console + parquet
\end{itemize}
\end{frame}

\begin{frame}{Conclusion}
\begin{itemize}
  \item Pipeline complet, robuste et extensible
  \item Améliorations : données réelles, modèles plus précis, streaming complet
\end{itemize}
\end{frame}

\end{document}
